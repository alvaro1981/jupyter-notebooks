%Declaracion del documento
\documentclass[a4paper,12pt]{article}
%Preambulo
%Las siguientes lineas solo son metadatos, aunque podermos utilizarlo mas adelante dentro del documento.
\title{Mi primer documento}
\date{2025-01-06}
\author{Alvaro Alarcon}
%Las siquiente dos lineas especifican paquetes de funciones no estandar que, en este caso, permiten al documento manejar adecuadamente textos en español.
\usepackage[spanish]{babel}
\usepackage[utf8]{inputenc}
% Inicio del documento propiamente dicho

\begin{document}
%Aqui comienza el cuerpo del articulo

\maketitle
\section{¡Hola, mundo!}
La energia es igual a la masa multiplicada por el cuadrado de la velocidad luz:

% Las formulas matematicas se escriben entre 4 simbolos de dolar (dos delante, y dos detras)

$$E = mc^2$$

%Asi se introduce una lista con dos items
\begin{itemize}
\item Primer elemento de la lista
\item Segundo elemento de la lista 
\end{itemize}
%Fin del documento
\end{document}


